\documentclass[12pt]{article}

\usepackage[T1]{fontenc}
\usepackage{lmodern}
\usepackage{textcomp}
\usepackage[latin1]{inputenc}
\usepackage[english]{babel}
\usepackage{amsmath}

\begin{document}

\tableofcontents

\newpage
\section{intro}

\section{reproducibility of research}

OK, so first of all I want to make it clear what I mean by reproducibility. In particular in relation to replicability of research. The thing is that both these terms are very often used interchangeably and that can sometimes lead to confusion. So here's the way I will use them here. 

We generally understand replication and reproducibility to be the golden standard of scientific research. Here's a definition I like "The confirmation of results and conclusions from one study obtained independently in another". This is from the editorial to a special issue of Science dedicated to "Data replicability and reproducibility". 

Here's another good one from the same issue, which is even more condensed: "[T]he independent verification of prior findings", although they are talking about reproducibility here. 

This of course goes back to Karl Popper and the Logic of scientific discovery and the concept of falsifiability. But I'm not going to get into the philosophy of science aspect here, all I want to do is make it clear what I'm going to talk about and so here's my definition of both terms:

When I talk of replication, I mean it in the broad sense relating to findings and conclusions. And within that there are several ways or levels of replication.
You can reask the question from scratch and attempt to confirm a finding or conclusion with a whole new experiment. You can redo an experiment - trying to follow the original methods as closely as possible. You can reanalyse the data from an experiment - using different methods, models what have you. And finally you can reproduce an analysis - using the same data, same methods and techniques. 

So it's essentially only the last type of replication that I'm talking about here. Sometimes these are also called "pure replications" or "close replications", but whatever you want to call them, that is what I'm talking about. 

So what I want to talk about are some simple and accessible tools and techniques and rules to follow to ensure that your work is reproducible in this sense. All of it actually also applies to level three here - that's essentially only a requirement to have open data access. Which is a slightly different issue if only because there might be confidentiality issues, but generally that's more straightforward. 


\section{Show your work}
So the sort of reproducibility that I'm talking about is about ``showing your work''. And yes, that can sometimes be the most annoying thing in the world. As in this example here, where a user on imgur vented his annoyance at his maths teacher who wanted him to show his work by taking it to a really extreme conclusion. I'm not sure if you can see this very well, this starts with ``Neurons travel through synapses, finger muscles move to grasp writing utensil'' and this whole page is just to describe the work that went into writing his name on the homework. And apparently this is followed by 44 more pages like this one. 

So obviously this is a bit of an extreme example, but my point is that honestly yes, ensuring your analysis is really reproducible can be quite annoying, and it can seem pointless and unnecessary.  BUT it is worth the hassle in the end. And it will be less of a hassle if you plan systematically on doing it from the start instead of thinking about it retrospecively and trying to fix it after the fact. And furthermore that doing truly reproducible data analysis has become easier and easier with a variety of tools not only being available, but being accessible, free and relatively easy to use. 

\section{RStudio}
OK, so I won't overdo it with trying to sell R as a data analysis tool. I think that's wat I put down under intended audience anyway, right, you have to think R is the best one ever? Right? So if you're here I'll assume you are already sold. But just in case, here is a chart from a recent survey of data scientists, I believe 1,200 of them were involved in this survey, and here are the proportions that use each of these programmes as their primary modelling and analytics tool. Now bear in mind these are data scientists, so if we look at academics more generally the picture is a bit different. These are from Google Scholar - the numbers of articles using each particular software tool, these are for 2014 which is the last complete year, and you can see that SPSS is still firmly in the lead, with R just trailing behind SAS in second place. 

But I think a massive part of this rise, that is also one that I think is set to continue, is the development of RStudio launched in 2011, which has quickly become the most popular IDE for R. Integrated Development Environment. And if it is not the most popular it is definitely the one with the highest rise in popularity. And with very good reason, not only did it fill a massive niche in the market, but it has also had an amazing level of development in the subsequent years. 

Of course there are still some old school coders who will use EMACS or some other favourite text editor and sending their code to the console, but RStudio has really made a huge difference in how accessible R has become, and how familiar the working environment is to the average user. Because here is how R used to look like, this is the R console in its most basic format. And this is what it looks like now with RStudio. 
So like R itself it is open source and free, it runs on the all the main operating systems. And it integrates the console and the text editor, but additionally also the plotting window, there is a data viewer as well, and a bunch of other tabs that allow you to view the folder tree, the project history, directly access help files and so on. So maybe lets just try having a look at what it looks like. 

Additionally - and this is not directly relevant for the topic of reproducible research, but in a way it kind of is. And that is RStudio Server. The server essentially allows you to run R and Rstudio remotely. And access it via your browser. This only works on a linux machines, although of course you can also run it in virtual box on a windows machine. Anyway, the point is you can run R from your browser, which means from a laptop that doesn't have R installed, or on a tablet or whatever. And I've also seen it used in courses for example, like a seminar or workshop at a university where instead of installing R on a bunch of machines, the organizers can simply set up a single machine as the server and have all the participants access it though their browsers. So just to show you - because I'm not allowed to install stuff on this machine here, but I have r server running at home on my laptop, we can try to access it from here. Unless someone's broken in and stolen it that is. OK, so here we have it: 

\section{git/Github}
OK, so version control. Does anyone here use git or github? Yeah, it's not the most accessible of softwares, for sure. OK, so git is a "distributed version control system for software development. What this basically means is that it was developed by software developers, coders if you like, for software developers. The idea is that when you are writing code, stuff works to a point, and then you mess it up somehow and it can be tricky to get back to the previous state, when it was still working. And to complicate matters further, sometimes you are writing code with someone else, and they add something in their version, and you've added something to yours, and now you want to have both new features work in a new merged version, and you don't want to just be copy pasting stufff across, especially if it will break the code, and then you will have trouble again getting back to the working version from before. So like I said, this was written by developers for themselves and git is notorious for being pretty difficult to get a grip of. This xkcd comic exemplifies this attitude - it's a super cool system, but yeah, it gets complicated really quickly. Here is an example of a git diagram taken from an online tutorial. So yeah, it can be pretty tricky. 


But, there is a but. git is fully integrated into RStudio, which means actually it is a lot a lot easier to use. You do not have to memorise any shell commands, you just need to add a simple new routine to your workflow, all from the comfort of this new RStudio tab over here. 

OK, so I should probably make this distinction clear: git is the name of the software.(and there are others as well, but I'll venture to guess git is the most widely used.  GitHub is the online repository hosting service. Because you can totally use git only on your own machine. That simply means you only have a local repository where your project is stored, and all the version control. But by storing your repository on GitHub, you get a bunch of additional benefits which I'll talk about in a second.  


So why is this relevant for reproducible research. Three reasons: one is for your own sake. This is an more elaborate way of keeping all of your work documented. So using version control means you can always revert back to any point in the process of your analysis. 
Second reason is collaboration. Using version control makes it easier to have multiple people working on the same files, same datasets, same coding scripts, same manuscripts while ensuring everyone's contribution is properly recorded, the project files remain consistent and by providing a central platform for all of this to happen, including discussions of issues that might arise. 

Even if you are not collaborating on a project, a github repository is an great tool for dissemination of your work. Simply link to your public repository. This is going way beyond an appendix or supplement to your paper that some journals now allow you to publish on line. This means others are able to access your data, your code, and a bunch of other stuff that didn't make it into the final paper. Did you have to limit yourself to one full colour plot in your paper? Well all the other 50 of them and the code for how to recreate them can be stored in a github repository. Note also that git means you have an authenticated history of your project. I'm not sure if it has ever been used this way, but a git version history proves what you did and when, if anyone were to steal your work, this is one way of proving you did it first. 

Additionally - although this should not be your prime motivation - but github is also a form of backup. The last working version that you committed is stored in the cloud, so there's always that. But you should anyway have dedicated backup solutions in place. GitHub I think allows a maximum of 1GB storage, and single files cannot exceed 100MB. But like I said, it is an extra backup. 

Now you may not be convinced that this is worth it, perhaps you're happy with Dropbox, and that covers most of your version control, collaboration and backup needs. Which is fine, I'm not going to try to force you. I want to just show you an example of how it works. So lets go back to my Rstudio example running on the 'server'. I'll just open a project here. OK, here is the git pane, and it is currenlty empty - that means the current state of the project has been committed, this point in time in the development of the project is stored. Lets just see how it works, yeah, looks like it works. OK, now I know it works, I will commit my changes - thus creating a new point in the git tree, and now I will push it to the github repository. 

Now I will inadvertently mess it up, like this. And save the results. And do something else for a while, and only then I realise that my code is gone, it doesn't work anymore!? But I know it worked before, I know it worked at my last commit, So what I do is simply revert to the previous commit like this, and we're back, it works again!

OK, let's go have a look at it online - because it's a public repository, we anyone can check it out here. This is the repository on github. You can see the whole project folder and all the subfolders. Over here you can see the history of the commits, where you can go through them, and it will highlight the differences made between each commit. 

OK, so this brings me to the final point on my github list and that is calling it facebook for coders. I mean that is maybe a bit of a stretch, but there are some similarities. For one, it is very often possible to find individual users - if they have used a suitably recognizable username. So let's look for Robin Lovelace, who I was just working on a project with, so I know he is on here. OK, so his user name is actually spelled without a space, but I can still get to him like this. And here are all his public repositories, and we can have a look around - don't worry we can't change anything, but we can definitely snoop around a bit. So there if you want to engage in some code vouyerism, github is the place to go. 
There is also a more sinister aspect to this - as you would expect with something I'm comparing with gitgub, but that is the fact that these public repositories are all searcheable, including all the code in them. So this is a good one, BEGIN RSA PRIVATE KEY extension:key, and let's see the ones that were most recently indexed. And there you have it. Now I wouldn't know what to do with these, but there have been examples of abuse, for example Uber apparently had the private key for a database of their drivers up here and that got hacked. 

Now you may be wondering why would i want to get involved in something like this, its way too dangerous. Well, for one it's not, if you're aware that you are working in a public repository, then just don't upload sensitive information to it, including passwords or data that is meant to be confidential. 

But there is of course another option. And that is the option to have private repositories on GitHub. So if you open a free account, what will happen is you only have public repositories. Then there is of course a whole set of payment plans that allow you to have private repositories as well. So there is a micro plan for 7 dollars a month that lets you have 5 private repos. BUT, there is good news! If you are a student or a university employee, then you can have the micro plan for free. All you need to do is claim your discount using this friendly link here. So there, that should be useful to help you get over your shyness. But I would really encourage you to try to keep as much of your code public though, you never know what wonderful collaborations might ensue. And obviously be extremely careful with confidential information and the like.. 

\section{Literate Programming - knitr}

So the next important concept in reproducible research  I want to talk about , and one that is super easy with R and RStudio, is literate programming. So this concept was devised by Donald Knuth, who is the legend behind the TeX typesetting system amongst other things. So the basic idea is that your code should be human readable. Now of course strictly speaking your code will always be human readable, but the question is how frustrated do you want that human to be. Because of course technically any machine readable code can be read by a human, and they can eventually figure out what the code does. What they cannot see from the code is WHY. Similarly reading a analysis report also allows a human to understand what the analysis did, but it does not fully explain HOW. 

\end{document}